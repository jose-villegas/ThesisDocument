% arara: indent: {overwrite: on}
\documentclass[12pt,twoside]{report}
\usepackage[margin=2cm]{geometry}
\usepackage{amsmath,amsthm,amssymb}
\usepackage{thmtools}
\usepackage{tikz}
\usepackage[framemethod=TikZ]{mdframed}

\declaretheoremstyle
[
	spaceabove=0pt, spacebelow=0pt, headfont=\normalfont\bfseries,
	notefont=\mdseries, notebraces={(}{)}, headpunct={\newline}, headindent={},
	postheadspace={ }, postheadspace=4pt, bodyfont=\normalfont, qed=$\blacktriangle$,
	preheadhook={\begin{mdframed}[style=myframedstyle]},
	postfoothook=\end{mdframed},
]{mystyle}

\declaretheorem[style=mystyle,numberwithin=chapter,title=Exemplo]{example}
\mdfdefinestyle{myframedstyle}{%
	outermargin = 1.3cm , %
	leftmargin = 0pt , rightmargin = 0pt , %
	innerleftmargin = 5pt , innerrightmargin = 5pt , %
	innertopmargin = 5pt, innerbottommargin = 5pt , %
	backgroundcolor = blue!10 , %
	align = center , % align the environment itself (left, center, rigth)
	nobreak = true, % prevent a frame from splitting
	hidealllines = true , %
	topline = true , bottomline = true , %
	splittopskip = \topskip , splitbottomskip = 0pt , %
	skipabove = 0.5\baselineskip ,  skipbelow = 0.3\baselineskip}

\begin{document}
\section{Introduction}
 Lorem ipsum sed nulla id risus adipiscing vulputate.

 \begin{example}
 	Um consumidor financiou a compra de um veículo pagando 48 parcelas de \$800,00 mensais e a taxa de juros cobrada pela concessionária foi de 1,2\% a.m.. Qual era o valor à vista do automóvel adquirido?
 	\newline
 	\textbf{Solução:}
 	\newline
 	$PV = 800 \times \left[ \dfrac{1,012^{48}-1}{1,012^{48}\times 0,012} \right] \newline
 	PV = 800 \times \left[ \dfrac{0,772820}{0,021274} \right] \newline
 	PV = \$29.061,79$
 \end{example}

 Lorem ipsum sed nulla id risus adipiscing vulputate.
\end{document}
