\section*{Resumen}
La iluminación de escenas es fundamental para la generación de imágenes de alta calidad, esta provee inmersión, sensación de profundidad y realismo. La producción de iluminación realista comprende, entre muchas otras cosas, la inclusión de iluminación indirecta para la composición de iluminación global.

Existen varios algoritmos para el cálculo de la iluminación global de forma analítica, sin embargo el costo computacional de estos es alto y dependiente de la complejidad de la escena. Esto hace estas soluciones poco flexibles o simplemente inadecuadas para aplicaciones en tiempo real, altamente interactivas o de considerable complejidad geométrica. 

Con el incremento de la capacidad de cómputo en las unidades de procesamiento gráfico modernas también ha aumentado el interés por la inclusión de fenómenos de iluminación global en aplicaciones en tiempo real. De esto han surgido una variada cantidad de aproximaciones explotando características del hardware de procesamiento gráfico y los recursos disponibles en el pipeline de renderizado. Estos algoritmos mantienen cierto grado de coherencia o convergen con la solución analítica al problema de iluminación global. 

Este trabajo tiene como enfoque principal el renderizado de iluminación global en tiempo real, por esto se realiza una investigación de estos algoritmos y se presenta el desarrollo de una aplicación con iluminación global utilizando trazado de vóxeles y conos.
\paragraph{Palabras Clave:}
Iluminación global, vóxeles, trazado de conos, iluminación indirecta, unidad de cómputo gráfico, renderizado en tiempo real.

\newpage
\section*{Agradecimientos}

Al profesor Esmitt Ramírez, por su apoyo y dirección, sin el cual este trabajo de investigación se hubiese llevado a cabo.

A mi mamá, Marina Villegas, por su apoyo y soporte incondicional en todos los aspectos de mi vida y por ser mi mejor amiga. 

A mi hermano menor Daniel De La Cruz, por su colaboración y ayuda para hacer este trabajo posible e inspirarme a ser un buen ejemplo para él.

A todos mis amigos por estar siempre atentos a este trabajo y su progreso el cual me mantuvo enfocado durante la realización de este trabajo. Y además por ser los causantes de múltiples distracciones cuando las necesitaba y no necesitaba.