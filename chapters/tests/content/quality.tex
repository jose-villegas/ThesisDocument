\section{Estudio de Calidad de Imagen}

En esta sección se estudiaron configuraciones que afectan la calidad de imagen resultante. Para el estudio de diferencia entre imágenes se utilizó el software PerceptualDiff basado en el trabajo de Hector Yee et al. en 2001 \cite{Yee:2001:SSV:383745.383748}, este programa utiliza un modelo computacional imitando al ojo humano para generar la diferencia perceptual entre dos imágenes.

\subsection{Composición Final de Imagen}
\label{subsec:final}
Todas las imágenes en esta sección fueron renderizadas con una resolución de pantalla de $1920x1080p$, con una resolución para la representación de vóxeles de $512^3$ vóxeles y con un factor de longitud de marcha del cono de $0.5$. La representación de vóxeles solo contiene iluminación directa.

\begin{figure}[H]
	\centering
	\begin{subfigure}[t]{.49\linewidth}
		\centering
		\captionsetup{justification=centering}
		\caption*{Directa}
		\includegraphics[width=\linewidth]{media/finals/cornell_direct.png}
	\end{subfigure}%
	\hspace{0.01\textwidth}
	\begin{subfigure}[t]{.49\linewidth}
		\centering
		\caption*{Indirecta}
		\captionsetup{justification=centering}
		\includegraphics[width=\linewidth]{media/finals/cornell_indirect.png}
	\end{subfigure}%
	\par\smallskip
	\begin{subfigure}[t]{.49\linewidth}
		\centering
		\captionsetup{justification=centering}
		\includegraphics[width=\linewidth]{media/finals/cornell_ao.png}
		\caption*{Oclusión Ambiental}
	\end{subfigure}%
	\hspace{0.01\textwidth}
	\begin{subfigure}[t]{.49\linewidth}
		\centering
		\captionsetup{justification=centering}
		\includegraphics[width=\linewidth]{media/finals/cornell_gi.png}
		\caption*{Resultado: Directa, Indirecta y Oclusión Ambiental}
	\end{subfigure}%
	\caption{Composición para la escena Cornell Box.}
	\label{fig:cornell_final}
\end{figure}
\begin{figure}[H]
	\centering
	\begin{subfigure}[t]{.49\linewidth}
		\centering
		\captionsetup{justification=centering}
		% \caption*{Directa}
		\includegraphics[width=\linewidth]{media/finals/sibenik_direct.png}
	\end{subfigure}%
	\hspace{0.01\textwidth}
	\begin{subfigure}[t]{.49\linewidth}
		\centering
		% \caption*{Indirecta}
		\captionsetup{justification=centering}
		\includegraphics[width=\linewidth]{media/finals/sibenik_indirect.png}
	\end{subfigure}%
	\par\smallskip
	\begin{subfigure}[t]{.49\linewidth}
		\centering
		% \caption*{Oclusión Ambiental}
		\captionsetup{justification=centering}
		\includegraphics[width=\linewidth]{media/finals/sibenik_ao.png}
	\end{subfigure}%
	\hspace{0.01\textwidth}
	\begin{subfigure}[t]{.49\linewidth}
		\centering
		% \caption*{Directa + Indirecta + Oclusión Ambiental}
		\captionsetup{justification=centering}
		\includegraphics[width=\linewidth]{media/finals/sibenik_gi.png}
	\end{subfigure}%
	\caption{Composición para la escena Sibenik.}
	\label{fig:sibenik_final}
\end{figure}
\begin{figure}[H]
	\centering
	\begin{subfigure}[t]{.49\linewidth}
		\centering
		\captionsetup{justification=centering}
		% \caption*{Directa}
		\includegraphics[width=\linewidth]{media/finals/sponza_direct.png}
	\end{subfigure}%
	\hspace{0.01\textwidth}
	\begin{subfigure}[t]{.49\linewidth}
		\centering
		% \caption*{Indirecta}
		\captionsetup{justification=centering}
		\includegraphics[width=\linewidth]{media/finals/sponza_indirect.png}
	\end{subfigure}%
	\par\smallskip
	\begin{subfigure}[t]{.49\linewidth}
		\centering
		% \caption*{Oclusión Ambiental}
		\captionsetup{justification=centering}
		\includegraphics[width=\linewidth]{media/finals/sponza_ao.png}
	\end{subfigure}%
	\hspace{0.01\textwidth}
	\begin{subfigure}[t]{.49\linewidth}
		\centering
		% \caption*{Directa + Indirecta + Oclusión Ambiental}
		\captionsetup{justification=centering}
		\includegraphics[width=\linewidth]{media/finals/sponza_gi.png}
	\end{subfigure}%
	\caption{Composición para la escena Sponza.}
	\label{fig:sponza_final}
\end{figure}
\begin{figure}[H]
	\centering
	\begin{subfigure}[t]{.49\linewidth}
		\centering
		\captionsetup{justification=centering}
		% \caption*{Directa}
		\includegraphics[width=\linewidth]{media/finals/conf_direct.png}
	\end{subfigure}%
	\hspace{0.01\textwidth}
	\begin{subfigure}[t]{.49\linewidth}
		\centering
		% \caption*{Indirecta}
		\captionsetup{justification=centering}
		\includegraphics[width=\linewidth]{media/finals/conf_indirect.png}
	\end{subfigure}%
	\par\smallskip
	\begin{subfigure}[t]{.49\linewidth}
		\centering
		% \caption*{Oclusión Ambiental}
		\captionsetup{justification=centering}
		\includegraphics[width=\linewidth]{media/finals/conf_ao.png}
	\end{subfigure}%
	\hspace{0.01\textwidth}
	\begin{subfigure}[t]{.49\linewidth}
		\centering
		% \caption*{Directa + Indirecta + Oclusión Ambiental}
		\captionsetup{justification=centering}
		\includegraphics[width=\linewidth]{media/finals/conf_gi.png}
	\end{subfigure}%
	\caption{Composición para la escena Conference.}
	\label{fig:conf_final}
\end{figure}

\subsubsection{Iluminación Global de Vóxeles.}

En esta sección se demuestra la diferencia de imagen final al agregar iluminación global a la representación con vóxeles. La inclusión de iluminación global sobre los vóxeles nos permite aproximar el segundo rebote de luz. La diferencia entre imágenes fue procesada utilizando PerceptualDiff, los píxeles azules describen donde se encuentran las áreas con mayor diferencia al ojo humano.

\begin{figure}[H]
	\centering
	\begin{subfigure}[b]{.49\linewidth}
		\centering
		\captionsetup{justification=centering}
		\caption*{Directa, Indirecta y Oclusión Ambiental}
		\includegraphics[width=\linewidth]{media/finals/cornell_vgi.png}
	\end{subfigure}%
	\hspace{0.01\textwidth}
	\begin{subfigure}[b]{.49\linewidth}
		\centering
		\captionsetup{justification=centering}
		\caption*{Diferencia Perceptual}
		\includegraphics[width=\linewidth]{media/finals/cornell_vgi_diff.png}
	\end{subfigure}%
	\caption{Diferencia con respecto al resultado en la Figura \ref{fig:cornell_final} con iluminación global de vóxeles}
	\label{fig:cornell_vgi_diff}
\end{figure}
\begin{figure}[H]
	\centering
	\begin{subfigure}[b]{.49\linewidth}
		\centering
		\captionsetup{justification=centering}
		%\caption*{Sombreado e Iluminación Global de Vóxeles}
		\includegraphics[width=\linewidth]{media/finals/sibenik_vgi.png}
	\end{subfigure}%
	\hspace{0.01\textwidth}
	\begin{subfigure}[b]{.49\linewidth}
		\centering
		\captionsetup{justification=centering}
		%\caption*{Diferencia Perceptual}
		\includegraphics[width=\linewidth]{media/finals/sibenik_vgi_diff.png}
	\end{subfigure}%
	\caption{Diferencia con respecto al resultado en la Figura \ref{fig:sibenik_final} con iluminación global de vóxeles}
	\label{fig:sibenik_vgi_diff}
\end{figure}
\begin{figure}[H]
	\centering
	\begin{subfigure}[b]{.49\linewidth}
		\centering
		\captionsetup{justification=centering}
		%\caption*{Sombreado e Iluminación Global de Vóxeles}
		\includegraphics[width=\linewidth]{media/finals/sponza_vgi.png}
	\end{subfigure}%
	\hspace{0.01\textwidth}
	\begin{subfigure}[b]{.49\linewidth}
		\centering
		\captionsetup{justification=centering}
		%\caption*{Diferencia Perceptual}
		\includegraphics[width=\linewidth]{media/finals/sponza_vgi_diff.png}
	\end{subfigure}%
	\caption{Diferencia con respecto al resultado en la Figura \ref{fig:sponza_final} con iluminación global de vóxeles}
	\label{fig:sponza_vgi_diff}
\end{figure}
\begin{figure}[H]
	\centering
	\begin{subfigure}[b]{.49\linewidth}
		\centering
		\captionsetup{justification=centering}
		%\caption*{Sombreado e Iluminación Global de Vóxeles}
		\includegraphics[width=\linewidth]{media/finals/conf_vgi.png}
	\end{subfigure}%
	\hspace{0.01\textwidth}
	\begin{subfigure}[b]{.49\linewidth}
		\centering
		\captionsetup{justification=centering}
		%\caption*{Diferencia Perceptual}
		\includegraphics[width=\linewidth]{media/finals/conf_vgi_diff.png}
	\end{subfigure}%
	\caption{Diferencia con respecto al resultado en la Figura \ref{fig:conf_final} con iluminación global de vóxeles}
	\label{fig:conf_vgi_diff}
\end{figure}

\subsubsection{Resolución de la Representación en Vóxeles}

En esta sección se demuestra la diferencia de imagen final con distintas resoluciones para la representación en vóxeles. Todas las imágenes serán comparadas con los resultados de la sección \ref{subsec:final} donde se utiliza una resolución para los volúmenes de $512^3$ vóxeles sin iluminación global de vóxeles. En las imágenes se puede apreciar como la diferencia aumenta a medida que se disminuye la resolución.

\begin{figure}[H]
	\centering
	\begin{subfigure}[b]{.49\linewidth}
		\centering
		\captionsetup{justification=centering}
		\caption*{Directa, Indirecta y Oclusión Ambiental}
		\includegraphics[width=\linewidth]{media/finals/cornell_gi_256.png}
		\caption*{$256^3$ vóxeles}
	\end{subfigure}%
	\hspace{0.01\textwidth}
	\begin{subfigure}[b]{.49\linewidth}
		\centering
		\captionsetup{justification=centering}
		\caption*{Diferencia Perceptual\\}
		\includegraphics[width=\linewidth]{media/finals/cornell_gi_256_diff.png}
		\caption*{}
	\end{subfigure}%
	\par\smallskip
	\begin{subfigure}[b]{.49\linewidth}
		\centering
		\captionsetup{justification=centering}
		\includegraphics[width=\linewidth]{media/finals/cornell_gi_128.png}
		\caption*{$128^3$ vóxeles}
	\end{subfigure}%
	\hspace{0.01\textwidth}
	\begin{subfigure}[b]{.49\linewidth}
		\centering
		\captionsetup{justification=centering}
		%\caption*{Diferencia Perceptual}
		\includegraphics[width=\linewidth]{media/finals/cornell_gi_128_diff.png}
		\caption*{}
	\end{subfigure}%
	\par\smallskip
	\begin{subfigure}[b]{.49\linewidth}
		\centering
		\captionsetup{justification=centering}
		\includegraphics[width=\linewidth]{media/finals/cornell_gi_64.png}
		\caption*{$64^3$ vóxeles}
	\end{subfigure}%
	\hspace{0.01\textwidth}
	\begin{subfigure}[b]{.49\linewidth}
		\centering
		\captionsetup{justification=centering}
		%\caption*{Diferencia Perceptual}
		\includegraphics[width=\linewidth]{media/finals/cornell_gi_64_diff.png}
		\caption*{}
	\end{subfigure}%
	\caption{Diferencia con respecto al resultado en la Figura \ref{fig:cornell_final} con distintas resoluciones para la representación en vóxeles.}
	\label{fig:cornell_gi_resdiff}
\end{figure}

\begin{figure}[H]
	\centering
	\begin{subfigure}[b]{.49\linewidth}
		\centering
		\captionsetup{justification=centering}
		%\caption*{Directa, Indirecta y Oclusión Ambiental}
		\includegraphics[width=\linewidth]{media/finals/sibenik_gi_256.png}
		%\caption*{$256^3$}
	\end{subfigure}%
	\hspace{0.01\textwidth}
	\begin{subfigure}[b]{.49\linewidth}
		\centering
		\captionsetup{justification=centering}
		%\caption*{Diferencia Perceptual\\}
		\includegraphics[width=\linewidth]{media/finals/sibenik_gi_256_diff.png}
		%\caption*{}
	\end{subfigure}%
	\par\smallskip
	\begin{subfigure}[b]{.49\linewidth}
		\centering
		\captionsetup{justification=centering}
		\includegraphics[width=\linewidth]{media/finals/sibenik_gi_128.png}
		%\caption*{$128^3$}
	\end{subfigure}%
	\hspace{0.01\textwidth}
	\begin{subfigure}[b]{.49\linewidth}
		\centering
		\captionsetup{justification=centering}
		%\caption*{Diferencia Perceptual}
		\includegraphics[width=\linewidth]{media/finals/sibenik_gi_128_diff.png}
		%\caption*{}
	\end{subfigure}%
	\par\smallskip
	\begin{subfigure}[b]{.49\linewidth}
		\centering
		\captionsetup{justification=centering}
		\includegraphics[width=\linewidth]{media/finals/sibenik_gi_64.png}
		%\caption*{$64^3$}
	\end{subfigure}%
	\hspace{0.01\textwidth}
	\begin{subfigure}[b]{.49\linewidth}
		\centering
		\captionsetup{justification=centering}
		%\caption*{Diferencia Perceptual}
		\includegraphics[width=\linewidth]{media/finals/sibenik_gi_64_diff.png}
		%\caption*{}
	\end{subfigure}%
	\caption{Diferencia con respecto al resultado en la Figura \ref{fig:sibenik_final} con distintas resoluciones para la representación en vóxeles.}
	\label{fig:sibenik_gi_resdiff}
\end{figure}

\begin{figure}[H]
	\centering
	\begin{subfigure}[b]{.49\linewidth}
		\centering
		\captionsetup{justification=centering}
		%\caption*{Directa, Indirecta y Oclusión Ambiental}
		\includegraphics[width=\linewidth]{media/finals/sponza_gi_256.png}
		%\caption*{$256^3$}
	\end{subfigure}%
	\hspace{0.01\textwidth}
	\begin{subfigure}[b]{.49\linewidth}
		\centering
		\captionsetup{justification=centering}
		%\caption*{Diferencia Perceptual\\}
		\includegraphics[width=\linewidth]{media/finals/sponza_gi_256_diff.png}
		%\caption*{}
	\end{subfigure}%
	\par\smallskip
	\begin{subfigure}[b]{.49\linewidth}
		\centering
		\captionsetup{justification=centering}
		\includegraphics[width=\linewidth]{media/finals/sponza_gi_128.png}
		%\caption*{$128^3$}
	\end{subfigure}%
	\hspace{0.01\textwidth}
	\begin{subfigure}[b]{.49\linewidth}
		\centering
		\captionsetup{justification=centering}
		%\caption*{Diferencia Perceptual}
		\includegraphics[width=\linewidth]{media/finals/sponza_gi_128_diff.png}
		%\caption*{}
	\end{subfigure}%
	\par\smallskip
	\begin{subfigure}[b]{.49\linewidth}
		\centering
		\captionsetup{justification=centering}
		\includegraphics[width=\linewidth]{media/finals/sponza_gi_64.png}
		%\caption*{$64^3$}
	\end{subfigure}%
	\hspace{0.01\textwidth}
	\begin{subfigure}[b]{.49\linewidth}
		\centering
		\captionsetup{justification=centering}
		%\caption*{Diferencia Perceptual}
		\includegraphics[width=\linewidth]{media/finals/sponza_gi_64_diff.png}
		%\caption*{}
	\end{subfigure}%
	\caption{Diferencia con respecto al resultado en la Figura \ref{fig:sponza_final} con distintas resoluciones para la representación en vóxeles.}
	\label{fig:sponza_gi_resdiff}
\end{figure}

\begin{figure}[H]
	\centering
	\begin{subfigure}[b]{.49\linewidth}
		\centering
		\captionsetup{justification=centering}
		%\caption*{Directa, Indirecta y Oclusión Ambiental}
		\includegraphics[width=\linewidth]{media/finals/conf_gi_256.png}
		%\caption*{$256^3$}
	\end{subfigure}%
	\hspace{0.01\textwidth}
	\begin{subfigure}[b]{.49\linewidth}
		\centering
		\captionsetup{justification=centering}
		%\caption*{Diferencia Perceptual\\}
		\includegraphics[width=\linewidth]{media/finals/conf_gi_256_diff.png}
		%\caption*{}
	\end{subfigure}%
	\par\smallskip
	\begin{subfigure}[b]{.49\linewidth}
		\centering
		\captionsetup{justification=centering}
		\includegraphics[width=\linewidth]{media/finals/conf_gi_128.png}
		%\caption*{$128^3$}
	\end{subfigure}%
	\hspace{0.01\textwidth}
	\begin{subfigure}[b]{.49\linewidth}
		\centering
		\captionsetup{justification=centering}
		%\caption*{Diferencia Perceptual}
		\includegraphics[width=\linewidth]{media/finals/conf_gi_128_diff.png}
		%\caption*{}
	\end{subfigure}%
	\par\smallskip
	\begin{subfigure}[b]{.49\linewidth}
		\centering
		\captionsetup{justification=centering}
		\includegraphics[width=\linewidth]{media/finals/conf_gi_64.png}
		%\caption*{$64^3$}
	\end{subfigure}%
	\hspace{0.01\textwidth}
	\begin{subfigure}[b]{.49\linewidth}
		\centering
		\captionsetup{justification=centering}
		%\caption*{Diferencia Perceptual}
		\includegraphics[width=\linewidth]{media/finals/conf_gi_64_diff.png}
		%\caption*{}
	\end{subfigure}%
	\caption{Diferencia con respecto al resultado en la Figura \ref{fig:conf_final} con distintas resoluciones para la representación en vóxeles.}
	\label{fig:conf_gi_resdiff}
\end{figure}

\subsubsection{Factor de Longitud de Marcha del Cono}

El factor de longitud de marcha del cono afecta tanto el rendimiento como la calidad visual final. Valores mayores a $1.0$ no son recomendados ya que saltarían vóxeles constantemente durante el trazado. En esta sección se compara contra el factor $0.5$ en la sección \ref{subsec:final} con valores de $1.0$ y $2.5$.


\begin{figure}[H]
	\centering
	\begin{subfigure}[b]{.49\linewidth}
		\centering
		\captionsetup{justification=centering}
		\caption*{Directa, Indirecta y Oclusión Ambiental}
		\includegraphics[width=\linewidth]{{media/finals/cornell_gi_s1.0}.png}
		\caption*{Factor de longitud de marcha: $1.0$}
	\end{subfigure}%
	\hspace{0.01\textwidth}
	\begin{subfigure}[b]{.49\linewidth}
		\centering
		\captionsetup{justification=centering}
		\caption*{Diferencia Perceptual\\}
		\includegraphics[width=\linewidth]{{media/finals/cornell_gi_s1.0_diff}.png}
		\caption*{}
	\end{subfigure}%
	\par\smallskip
	\begin{subfigure}[b]{.49\linewidth}
		\centering
		\captionsetup{justification=centering}
		\includegraphics[width=\linewidth]{{media/finals/cornell_gi_s2.5}.png}
		\caption*{$2.5$}
	\end{subfigure}%
	\hspace{0.01\textwidth}
	\begin{subfigure}[b]{.49\linewidth}
		\centering
		\captionsetup{justification=centering}
		%\caption*{Diferencia Perceptual}
		\includegraphics[width=\linewidth]{{media/finals/cornell_gi_s2.5_diff}.png}
		\caption*{}
	\end{subfigure}%
	\caption{Diferencia con respecto al resultado en la Figura \ref{fig:cornell_final} utilizando distintos factores de longitud.}
	\label{fig:cornell_gi_sdiff}
\end{figure}

\begin{figure}[H]
	\centering
	\begin{subfigure}[b]{.49\linewidth}
		\centering
		\captionsetup{justification=centering}
		%\caption*{Directa, Indirecta y Oclusión Ambiental}
		\includegraphics[width=\linewidth]{{media/finals/sibenik_gi_s1.0}.png}
		%\caption*{$1.0$}
	\end{subfigure}%
	\hspace{0.01\textwidth}
	\begin{subfigure}[b]{.49\linewidth}
		\centering
		\captionsetup{justification=centering}
		%\caption*{Diferencia Perceptual\\}
		\includegraphics[width=\linewidth]{{media/finals/sibenik_gi_s1.0_diff}.png}
		%\caption*{}
	\end{subfigure}%
	\par\smallskip
	\begin{subfigure}[b]{.49\linewidth}
		\centering
		\captionsetup{justification=centering}
		\includegraphics[width=\linewidth]{{media/finals/sibenik_gi_s2.5}.png}
		%\caption*{$2.5$}
	\end{subfigure}%
	\hspace{0.01\textwidth}
	\begin{subfigure}[b]{.49\linewidth}
		\centering
		\captionsetup{justification=centering}
		%\caption*{Diferencia Perceptual}
		\includegraphics[width=\linewidth]{{media/finals/sibenik_gi_s2.5_diff}.png}
		%\caption*{}
	\end{subfigure}%
	\caption{Diferencia con respecto al resultado en la Figura \ref{fig:sibenik_final} utilizando distintos factores de longitud.}
	\label{fig:sibenik_gi_sdiff}
\end{figure}

\begin{figure}[H]
	\centering
	\begin{subfigure}[b]{.49\linewidth}
		\centering
		\captionsetup{justification=centering}
		%\caption*{Directa, Indirecta y Oclusión Ambiental}
		\includegraphics[width=\linewidth]{{media/finals/conf_gi_s1.0}.png}
		%\caption*{$1.0$}
	\end{subfigure}%
	\hspace{0.01\textwidth}
	\begin{subfigure}[b]{.49\linewidth}
		\centering
		\captionsetup{justification=centering}
		%\caption*{Diferencia Perceptual\\}
		\includegraphics[width=\linewidth]{{media/finals/conf_gi_s1.0_diff}.png}
		%\caption*{}
	\end{subfigure}%
	\par\smallskip
	\begin{subfigure}[b]{.49\linewidth}
		\centering
		\captionsetup{justification=centering}
		\includegraphics[width=\linewidth]{{media/finals/conf_gi_s2.5}.png}
		%\caption*{$2.5$}
	\end{subfigure}%
	\hspace{0.01\textwidth}
	\begin{subfigure}[b]{.49\linewidth}
		\centering
		\captionsetup{justification=centering}
		%\caption*{Diferencia Perceptual}
		\includegraphics[width=\linewidth]{{media/finals/conf_gi_s2.5_diff}.png}
		%\caption*{}
	\end{subfigure}%
	\caption{Diferencia con respecto al resultado en la Figura \ref{fig:conf_final} utilizando distintos factores de longitud.}
	\label{fig:conf_gi_sdiff}
\end{figure}

\begin{figure}[H]
	\centering
	\begin{subfigure}[b]{.49\linewidth}
		\centering
		\captionsetup{justification=centering}
		%\caption*{Directa, Indirecta y Oclusión Ambiental}
		\includegraphics[width=\linewidth]{{media/finals/sponza_gi_s1.0}.png}
		%\caption*{$1.0$}
	\end{subfigure}%
	\hspace{0.01\textwidth}
	\begin{subfigure}[b]{.49\linewidth}
		\centering
		\captionsetup{justification=centering}
		%\caption*{Diferencia Perceptual\\}
		\includegraphics[width=\linewidth]{{media/finals/sponza_gi_s1.0_diff}.png}
		%\caption*{}
	\end{subfigure}%
	\par\smallskip
	\begin{subfigure}[b]{.49\linewidth}
		\centering
		\captionsetup{justification=centering}
		\includegraphics[width=\linewidth]{{media/finals/sponza_gi_s2.5}.png}
		%\caption*{$2.5$}
	\end{subfigure}%
	\hspace{0.01\textwidth}
	\begin{subfigure}[b]{.49\linewidth}
		\centering
		\captionsetup{justification=centering}
		%\caption*{Diferencia Perceptual}
		\includegraphics[width=\linewidth]{{media/finals/sponza_gi_s2.5_diff}.png}
		%\caption*{}
	\end{subfigure}%
	\caption{Diferencia con respecto al resultado en la Figura \ref{fig:sponza_final} utilizando distintos factores de longitud.}
	\label{fig:sponza_gi_sdiff}
\end{figure}

\subsection{Reflexión Especular y Factor de Longitud de Marcha}

El factor de longitud afecta especialmente la calidad de las reflexiones especulares finas. En la Figura \ref{fig:spec_reflex_comp1} se puede observar como empiezan a aparecer anomalías y vacíos en la reflexión especular a medida que se aumenta el factor de longitud.

\begin{figure}[H]
	\centering
	\includegraphics[width=\linewidth]{media/finals/test_s010.png}
	\caption{Demostración de reflexión especular con factor de longitud de marcha $0.1$ y volúmenes con resolución de $512^3$ vóxeles.}
	\label{fig:spec_reflec010}
\end{figure}
\begin{figure}[H]
	\centering
	\begin{subfigure}[b]{.49\linewidth}
		\centering
		\captionsetup{justification=centering}
		\includegraphics[width=\linewidth]{media/finals/test_s050.png}
		\caption*{Factor de longitud de marcha: $0.5$}
	\end{subfigure}%
	\hspace{0.01\textwidth}
	\begin{subfigure}[b]{.49\linewidth}
		\centering
		\captionsetup{justification=centering}
		\includegraphics[width=\linewidth]{media/finals/test_s050_diff.png}
		\caption*{Diferencia con la Figura \ref{fig:spec_reflec010}}
	\end{subfigure}%
\par\smallskip
	\begin{subfigure}[b]{.49\linewidth}
		\centering
		\captionsetup{justification=centering}
		\includegraphics[width=\linewidth]{media/finals/test_s100.png}
		\caption*{$1.0$}
	\end{subfigure}%
	\hspace{0.01\textwidth}
	\begin{subfigure}[b]{.49\linewidth}
		\centering
		\captionsetup{justification=centering}
		%\caption*{Diferencia Perceptual}
		\includegraphics[width=\linewidth]{media/finals/test_s100_diff.png}
		\caption*{}
	\end{subfigure}%
	\par\smallskip
	\begin{subfigure}[b]{.49\linewidth}
		\centering
		\captionsetup{justification=centering}
		\includegraphics[width=\linewidth]{media/finals/test_s250.png}
		\caption*{$2.5$}
	\end{subfigure}%
	\hspace{0.01\textwidth}
	\begin{subfigure}[b]{.49\linewidth}
		\centering
		\captionsetup{justification=centering}
		%\caption*{Diferencia Perceptual}
		\includegraphics[width=\linewidth]{media/finals/test_s250_diff.png}
		\caption*{}
	\end{subfigure}%
	\par\smallskip
	\begin{subfigure}[b]{.32\linewidth}
		\centering
		\includegraphics[width=\linewidth]{media/step_error1.png}
		\caption*{Bandas de Color}
	\end{subfigure}%
	\hspace{0.01\textwidth}
	\begin{subfigure}[b]{.32\linewidth}
		\centering
		\includegraphics[width=\linewidth]{media/step_error2.png}
		\caption*{Salto de Vóxeles.}
	\end{subfigure}%
	\hspace{0.01\textwidth}
	\begin{subfigure}[b]{.32\linewidth}
		\centering
		\includegraphics[width=\linewidth]{media/step_error3.png}
		\caption*{Rayado o \emph{Striping}}
	\end{subfigure}%
	\caption{Diferencia con respecto a la Figura \ref{fig:spec_reflec010} considerando distintos factores de longitud. En la última imagen se puede observar en detalle algunos problemas visuales que surgen al incrementar longitud de marcha del cono.}
	\label{fig:spec_reflex_comp1}
\end{figure}

\subsection{Apertura del Cono para Trazado de Sombras Suaves}

Una de las características del trazado de conos es que mientras mayor es la apertura del cono más rápido es el trazado (ver Figura \ref{fig:fine_cones}). Esto presenta una ventaja para el trazado de sombras suaves ya que a medida que se abre el cono de sombreado más suaves son las sombras resultantes como se observa en la Figura \ref{fig:soft_aperture}.

\begin{figure}[H]
	\centering
	\begin{subfigure}[b]{.49\linewidth}
		\centering
		\captionsetup{justification=centering}
		\caption*{Ángulo de apertura: 1 grado}
		\includegraphics[width=\linewidth]{media/finals/shadow_1.png}
	\end{subfigure}%
	\hspace{0.01\textwidth}
	\begin{subfigure}[b]{.49\linewidth}
		\centering
		\captionsetup{justification=centering}
		\caption*{5 grados}
		\includegraphics[width=\linewidth]{media/finals/shadow_5.png}
	\end{subfigure}%
	\par\smallskip
	\begin{subfigure}[b]{.49\linewidth}
		\centering
		\captionsetup{justification=centering}
		\caption*{15 grados}
		\includegraphics[width=\linewidth]{media/finals/shadow_15.png}
	\end{subfigure}%
	\hspace{0.01\textwidth}
	\begin{subfigure}[b]{.49\linewidth}
		\centering
		\captionsetup{justification=centering}
		\caption*{25 grados}
		\includegraphics[width=\linewidth]{media/finals/shadow_25.png}
	\end{subfigure}%
	\caption{Sombras suaves generadas bajo distintas aperturas del cono de sombreado.}
	\label{fig:soft_aperture}
\end{figure}

\subsubsection{Mapeo del Volumen de Visibilidad}
Durante el proceso de sombreado se almacena la visibilidad de cada vóxel en el volumen de visibilidad. Estos valores pueden ser utilizados para mapeo de sombras proyectando la posición del fragmento en espacio de textura para obtener un valor de oclusión aproximado. Debido a la baja resolución de la representación del volumen esta técnica provee sombras de baja calidad pero a un rendimiento más alto como se observa en la Figura \ref{fig:fine_cones}. En esta sección se mostraran las tres técnicas de sombreado implementadas: mapeo de sombras, sombras con trazado de conos, mapeo del volumen de visibilidad. Para el mapeo de sombras se utilizó un mapa con resolución de $512x512$ píxeles, la resolución de la representación en vóxeles fue de $512^3$ vóxeles y el factor de longitud de marcha fue $0.5$. Se utilizó la escena sandbox Plane Test con distintas primitivas de diferente complejidad geométrica y una fuente de luz direccional.

\begin{figure}[H]
	\centering
	\begin{subfigure}[b]{.49\linewidth}
		\centering
		\includegraphics[width=\linewidth]{media/shadow_mapping.png}
		\caption*{Mapeo de Sombras}
	\end{subfigure}%
	\hspace{0.01\textwidth}
	\begin{subfigure}[b]{.49\linewidth}
		\centering
		\includegraphics[width=\linewidth]{media/trace_shadow.png}
		\caption*{Trazado de Conos.}
	\end{subfigure}%
	\par\smallskip
	\begin{subfigure}[b]{.8\linewidth}
		\centering
		\includegraphics[width=\linewidth]{media/visibility_volume.png}
		\caption*{Volumen de Visibilidad.}
	\end{subfigure}%
	\caption{Distintas técnicas para la generación de sombras sobre objetos de distinta complejidad geométrica.}
	\label{fig:shadowing_techniques}
\end{figure}

Para la generación de sombras tanto el trazado de conos como el mapeo del volumen de visibilidad puede ocasionar errores sobre geometría suave como la esfera o el toro mientras que el mapeo de sombras no parece presentar estos problemas. En la Figura \ref{fig:shadowing_techniques} el mapeo del volumen de visibilidad produce considerables artefactos visuales para toda geometría no definida por ángulos rectos, como el icosaedro, el toro y la esfera haciendo su uso muy limitado, sin embargo para objetos como el cubo y el plano produce resultados similares a las otras técnicas implementadas.

\subsection{Materiales Emisivos}

Nuestra implementación permite la aproximación de materiales emisivos con la adición de otro volumen durante el proceso de voxelización. Estos materiales pueden ser utilizados para simular luces de área. En esta sección se puede observar estos materiales en acción.

\begin{figure}[H]
	\centering
	\includegraphics[width=.9\linewidth]{media/finals/area_teapot.png}
	\caption{Modelo Utah Teapot emisivo junto a otros objetos precargados.}
	\label{fig:areapot}
\end{figure}
En la Figura \ref{fig:areapot} se observa la escena sandbox Cornell Box Vacio sin fuentes de luz tradicionales, el modelo de Utah Teapot tiene un material emisivo blanco que ilumina el resto de los objetos cercanos. La oclusión ambiental esta desactivada para esta imagen. Las sombras debajo de los objetos son generadas de forma natural por el trazado de conos contra vóxeles.
\begin{figure}[H]
	\centering
	\includegraphics[width=.9\linewidth]{media/finals/area_sponza.png}
	\caption{Escena Sponza iluminada por varios materiales emisivos de distintos colores.}
	\label{fig:areasponza}
\end{figure}
En la Figura \ref{fig:areasponza} se puede observar en la escena completa Sponza mezcla de colores por distintos materiales emisivos y reflexión especular de estos materiales.
\begin{figure}[H]
	\centering
	\includegraphics[width=.9\linewidth]{media/finals/area_shadows.png}
	\caption{Sombras suaves generadas por materiales emisivos.}
	\label{fig:areashadows}
\end{figure}
En la Figura \ref{fig:areashadows} se puede observar en la escena sandbox LightBox la generación de sombras suaves y su direccionalidad con respecto al material emisivo en la esfera.

\begin{figure}[H]
	\centering
	\includegraphics[width=.9\linewidth]{media/finals/fine_emissive.png}
	\caption{Detalles finos en materiales con emisión.}
	\label{fig:fine_emissive}
\end{figure}
En la Figura \ref{fig:fine_emissive} se puede observar en la escena completa Sibenik los detalles de emision para materiales con texturizado, el color resultante en la pared izquierda y derecha son distintos.
\subsection{Defectos o Artefactos Visuales} % (fold)
\label{sub:defectos_o_artefactos_visuales}

\begin{figure}[H]
	\centering
	\centering
	\begin{subfigure}[b]{0.623\linewidth}
		\centering
		\captionsetup{justification=centering}
		\includegraphics[width=\linewidth]{media/leaking.png}
	\end{subfigure}%
	\hspace{0.01\textwidth}
	\begin{subfigure}[b]{0.357\linewidth}
		\centering
		\captionsetup{justification=centering}
		\includegraphics[width=\linewidth]{media/leaking_detail.png}
	\end{subfigure}%
	\caption{Fuga de luz o \emph{light-leaking} en escena Sibenik.}
	\label{fig:leaking}
\end{figure}

En la Figura \ref{fig:leaking} se puede observar un defecto visual común del algoritmo. Este se puede producirse por defectos de voxelización cuando se falta o excede la cantidad de vóxeles. Esto sucede usualmente por errores en la geometría de la malla. Sin embargo también puede ser causado por el mismo algoritmo de voxelización por problemas de precisión en el proceso de rasterización al generar vóxeles en los bordes de la geometría poligonal. Para este caso una forma de reducir este problema es utilizando \ac{MSAA} \cite{gpuvoxelization}. Los vóxeles faltantes o sobrantes pueden causar que se iluminen vóxeles que no deberían iluminarse durante el proceso de sombreado y/o de iluminación global de vóxeles.

Otra condición en la que esto puede suceder es durante el trazado de conos con vóxeles cuando el trazado salta algunas de las muestras durante el recorrido ya sea dentro del nivel más alto de detalle o por falta de precisión en la interpolación de los niveles de mipmap. La utilización de vóxeles anisotrópicos reduce en gran manera esta posibilidad.

\begin{figure}[H]
	\centering
	\centering
	\begin{subfigure}[b]{0.623\linewidth}
		\centering
		\captionsetup{justification=centering}
		\includegraphics[width=\linewidth]{media/color_banding.png}
	\end{subfigure}%
	\hspace{0.01\textwidth}
	\begin{subfigure}[b]{0.357\linewidth}
		\centering
		\captionsetup{justification=centering}
		\includegraphics[width=\linewidth]{media/color_banding_detail.png}
	\end{subfigure}%
	\caption{Bandas de colores o \emph{color banding}.}
	\label{fig:banding}
\end{figure}

La aparición de bandas de colores sucede por problemas de muestreo durante el trazado de conos. La distancia entre muestras hace produce estas bandas súbitas producto de la interpolación trilineal entre un mismo nivel de detalle o interpolación cuadrilineal entre varios. Una forma de reducir esto es utilizando un cono de menor apertura sacrificando reflectividad en el caso de conos especulares o reduciendo la longitud de marcha del cono a cambio de menor rendimiento.

\subsection{Comparación} % (fold)
Para realizar una comparación de calidad se utilizó el algoritmo \ac{CLPV} el cual tiene resultados en rendimiento similares a nuestra implementación. Esta aproximación solo comprende iluminación indirecta difusa. En la figura \ref{fig:comp_quality} se puede observar que los bloques debajo de las columnas reflejan las áreas iluminadas del pasillo superior en la escena completa Sponza por reflexión especular. La reflexión difusa de las telas sobre las columnas es un poco más extensa en \ac{CLPV} llegando causando mezclado de colores hasta en los arcos entre las columnas.
\label{sub:comparacion}
\begin{figure}[H]
	\centering
	\begin{subfigure}[b]{0.49\linewidth}
		\centering
		\captionsetup{justification=centering}
		\includegraphics[width=\linewidth]{media/vct.png}
		\caption{}
	\end{subfigure}%
	\hspace{0.01\textwidth}
	\begin{subfigure}[b]{0.49\linewidth}
		\centering
		\captionsetup{justification=centering}
		\includegraphics[width=\linewidth]{media/lpv.png}
		\caption{}
	\end{subfigure}%
	\caption{Comparación de nuestra implementación (a) con algoritmo para el cálculo de iluminación indirecta con \acl{CLPV} (b).}
	\label{fig:comp_quality}
\end{figure}

% subsection comparacion (end)
% subsection defectos_o_artefactos_visuales (end)