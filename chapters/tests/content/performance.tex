\section{Estudio de Rendimiento}

Para el estudio de rendimiento de utilizo la cantidad de tiempo entre rutinas relevantes a cada prueba en milisegundos. Para obtener tiempos precisos en GPU se utilizó el software GPU PerfStudio y análisis de frame. Este software nos permite observar los detalles en el pipeline de renderizado y tiempo total de cada llamada de dibujo o \emph{draw call} al API grafico OpenGL

\subsection{Prueba Base}
La prueba base de rendimiento comprende todos los pasos del algoritmo sin trazado de sombras o modificaciones sobre la escena estática. Para esta prueba se colocó la aplicación en actualización forzosa de tal manera que todos los pasos del algoritmo se realicen por frame. También se varían tres aspectos de la aplicación: resolución de pantalla, dimensión de la representación en vóxeles y factor de longitud de marcha del cono.

Esta prueba utiliza todas las escenas completas. La configuración del escenario comprende una luz direccional con mapeado de sombras, y tres modelos precargados. Los modelos utilizados son Stanford Happy Buddha, Stanford Dragon y Stanford Bunny, estos modelos fueron seleccionados por su complejidad geométrica. Cada modelo tiene su propio material con un cono especular de apertura 45, 27 y 9 grados respectivamente. La cámara en escena es colocada de tal forma que todos los fragmentos visibles formen parte del trazado de conos. 

\begin{table}[h]
\centering
\begin{tabular}{llllllllll}
\rot{Escena}                    & \rot{Voxelización Estática} & \rot{Limpieza de Vóxeles Dinámicos} & \rot{Voxelización Dinámica} & \rot{Sombreado de Vóxeles} & \rot{Mipmapping Direccional} & \rot{Iluminación Global de Vóxeles} & \rot{Mipmapping Direccional} & \rot{Trazado de Conos con Vóxeles} & \rot{Tiempo Dinámico}    \\ \hline
\multicolumn{1}{|l|}{Sibenik}     & \multicolumn{1}{l|}{1.80}     & 0.58                                  & \multicolumn{1}{l|}{2.11}     & 0.95                         & \multicolumn{1}{l|}{1.39}      & 3.88                                  & \multicolumn{1}{l|}{1.38}      & \multicolumn{1}{l|}{7.31}            & \multicolumn{1}{l|}{17.60} \\
\multicolumn{1}{|l|}{Cornell Box} & \multicolumn{1}{l|}{0.51}     & 0.78                                  & \multicolumn{1}{l|}{1.30}     & 1.33                         & \multicolumn{1}{l|}{1.38}      & 8.41                                  & \multicolumn{1}{l|}{1.37}      & \multicolumn{1}{l|}{7.23}            & \multicolumn{1}{l|}{21.81} \\
\multicolumn{1}{|l|}{Conference}  & \multicolumn{1}{l|}{46.04}    & 0.56                                  & \multicolumn{1}{l|}{1.52}     & 0.86                         & \multicolumn{1}{l|}{1.38}      & 3.23                                  & \multicolumn{1}{l|}{1.37}      & \multicolumn{1}{l|}{7.50}            & \multicolumn{1}{l|}{16.42} \\
\multicolumn{1}{|l|}{Sponza}      & \multicolumn{1}{l|}{11.29}    & 0.60                                  & \multicolumn{1}{l|}{2.03}     & 1.13                         & \multicolumn{1}{l|}{1.37}      & 5.44                                  & \multicolumn{1}{l|}{1.38}      & \multicolumn{1}{l|}{7.01}            & \multicolumn{1}{l|}{18.97} \\ \hline
\end{tabular}
\captionsetup{justification=centering}
\caption{Rendimiento de todas las partes del algoritmo en distintas escenas utilizando volumenes de resolución $256^3$, factor de longitud de marcha de $0.5$ y resolución de pantalla de $1280x720$. Todos los tiempos en milisegundos.}
\label{tab:performance_base}
\end{table}

\begin{table}[H]
\centering
\begin{tabular}{lllllllllll}
\rot{Resolucion de Volumenes}                 & \rot{Escena}                   & \rot{Voxelización Estática} & \rot{Limpieza de Vóxeles Dinámicos} & \rot{Voxelización Dinámica} & \rot{Sombreado de Vóxeles} & \rot{Mipmapping Direccional} & \rot{Iluminación Global de Vóxeles} & \rot{Mipmapping Direccional} & \rot{Trazado de Conos con Vóxeles} & \rot{Tiempo Dinámico}    \\ \hline
\multicolumn{1}{|l|}{\multirow{4}{*}{\rotv{$64^3$}}}  & \multicolumn{1}{l|}{Sibenik}     & \multicolumn{1}{l|}{3.00}     & 0.01                                  & \multicolumn{1}{l|}{9.17}     & 0.04                         & \multicolumn{1}{l|}{0.11}      & 0.12                                  & \multicolumn{1}{l|}{0.10}      & \multicolumn{1}{l|}{5.17}            & \multicolumn{1}{l|}{14.74} \\
\multicolumn{1}{|l|}{}                     & \multicolumn{1}{l|}{Cornell Box} & \multicolumn{1}{l|}{0.07}     & 0.02                                  & \multicolumn{1}{l|}{2.13}     & 0.06                         & \multicolumn{1}{l|}{0.11}      & 0.29                                  & \multicolumn{1}{l|}{0.11}      & \multicolumn{1}{l|}{4.79}            & \multicolumn{1}{l|}{7.51}  \\
\multicolumn{1}{|l|}{}                     & \multicolumn{1}{l|}{Conference}  & \multicolumn{1}{l|}{39.68}    & 0.01                                  & \multicolumn{1}{l|}{5.98}     & 0.03                         & \multicolumn{1}{l|}{0.11}      & 0.10                                  & \multicolumn{1}{l|}{0.11}      & \multicolumn{1}{l|}{5.38}            & \multicolumn{1}{l|}{11.72} \\
\multicolumn{1}{|l|}{}                     & \multicolumn{1}{l|}{Sponza}      & \multicolumn{1}{l|}{22.87}    & 0.01                                  & \multicolumn{1}{l|}{13.93}    & 0.05                         & \multicolumn{1}{l|}{0.17}      & 0.13                                  & \multicolumn{1}{l|}{0.17}      & \multicolumn{1}{l|}{5.49}            & \multicolumn{1}{l|}{19.97} \\ \hline
\multicolumn{1}{|l|}{\multirow{4}{*}{\rotv{$128^3$}}} & \multicolumn{1}{l|}{Sibenik}     & \multicolumn{1}{l|}{2.17}     & 0.08                                  & \multicolumn{1}{l|}{3.93}     & 0.17                         & \multicolumn{1}{l|}{0.26}      & 0.62                                  & \multicolumn{1}{l|}{0.26}      & \multicolumn{1}{l|}{2.17}            & \multicolumn{1}{l|}{3.93}  \\
\multicolumn{1}{|l|}{}                     & \multicolumn{1}{l|}{Cornell Box} & \multicolumn{1}{l|}{0.16}     & 0.10                                  & \multicolumn{1}{l|}{1.38}     & 0.28                         & \multicolumn{1}{l|}{0.26}      & 1.62                                  & \multicolumn{1}{l|}{0.26}      & \multicolumn{1}{l|}{0.16}            & \multicolumn{1}{l|}{1.38}  \\
\multicolumn{1}{|l|}{}                     & \multicolumn{1}{l|}{Conference}  & \multicolumn{1}{l|}{39.19}    & 0.07                                  & \multicolumn{1}{l|}{2.47}     & 0.15                         & \multicolumn{1}{l|}{0.26}      & 0.54                                  & \multicolumn{1}{l|}{0.26}      & \multicolumn{1}{l|}{39.19}           & \multicolumn{1}{l|}{2.47}  \\
\multicolumn{1}{|l|}{}                     & \multicolumn{1}{l|}{Sponza}      & \multicolumn{1}{l|}{15.15}    & 0.08                                  & \multicolumn{1}{l|}{4.00}     & 0.20                         & \multicolumn{1}{l|}{0.26}      & 0.82                                  & \multicolumn{1}{l|}{0.26}      & \multicolumn{1}{l|}{15.15}           & \multicolumn{1}{l|}{4.00}  \\ \hline
\multicolumn{1}{|l|}{\multirow{4}{*}{\rotv{$512^3$}}} & \multicolumn{1}{l|}{Sibenik}     & \multicolumn{1}{l|}{2.36}     & 4.51                                  & \multicolumn{1}{l|}{1.35}     & 6.06                         & \multicolumn{1}{l|}{10.92}     & 23.23                                 & \multicolumn{1}{l|}{10.91}     & \multicolumn{1}{l|}{8.57}            & \multicolumn{1}{l|}{65.55} \\
\multicolumn{1}{|l|}{}                     & \multicolumn{1}{l|}{Cornell Box} & \multicolumn{1}{l|}{2.17}     & 6.08                                  & \multicolumn{1}{l|}{1.80}     & 7.54                         & \multicolumn{1}{l|}{10.70}     & 41.14                                 & \multicolumn{1}{l|}{10.87}     & \multicolumn{1}{l|}{7.87}            & \multicolumn{1}{l|}{86.00} \\
\multicolumn{1}{|l|}{}                     & \multicolumn{1}{l|}{Conference}  & \multicolumn{1}{l|}{47.44}    & 4.46                                  & \multicolumn{1}{l|}{1.28}     & 5.63                         & \multicolumn{1}{l|}{11.57}     & 18.70                                 & \multicolumn{1}{l|}{11.35}     & \multicolumn{1}{l|}{8.56}            & \multicolumn{1}{l|}{61.56} \\
\multicolumn{1}{|l|}{}                     & \multicolumn{1}{l|}{Sponza}      & \multicolumn{1}{l|}{9.05}     & 4.55                                  & \multicolumn{1}{l|}{1.39}     & 6.87                         & \multicolumn{1}{l|}{11.16}     & 28.09                                 & \multicolumn{1}{l|}{10.79}     & \multicolumn{1}{l|}{7.64}            & \multicolumn{1}{l|}{70.49} \\ \hline
\end{tabular}
\captionsetup{justification=centering}
\caption{Rendimiento en contraste con la tabla \ref{tab:performance_base} considerando distintas resoluciones para la representación en vóxeles, todos los pasos del algoritmo se ven afectados por este parámetro.}
\label{tab:performance_depth}
\end{table}
\begin{table}[H]
\centering
\begin{tabular}{|l|l|l|}
\hline
Resolución Pantalla  & 1920x1080                    & 1280x720                     \\ \hline
Escena      & Trazado de Conos con Vóxeles & Trazado de Conos con Vóxeles \\ \hline
Sibenik     & 14.30                        & 7.31                         \\
Cornell Box & 15.17                        & 7.23                         \\
Conference  & 16.27                        & 7.50                         \\
Sponza      & 16.69                        & 7.01                         \\ \hline
\end{tabular}
\captionsetup{justification=centering}
\caption{Rendimiento con una mayor resolución de pantalla. Esto solo afecta el trazado de conos con vóxeles.}
\label{tab:performance_display}
\end{table}
\begin{table}[H]
\centering
\begin{tabular}{|l|l|l|l|l|}
\hline
Longitud de Marcha & 0.1         & 0.25        & 0.5       & 2.5       \\ \hline
Escena             & \multicolumn{4}{l|}{Trazado de Conos con Vóxeles} \\ \hline
Sibenik            & 29.55       & 12.73       & 7.31      & 2.67      \\
Cornell Box        & 26.80       & 11.57       & 7.23      & 2.68      \\
Conference         & 29.73       & 12.93       & 7.50      & 2.85      \\
Sponza             & 29.81       & 12.92       & 7.01      & 2.81      \\ \hline\end{tabular}
\captionsetup{justification=centering}
\caption{Rendimiento con distintas configuraciones de la longitud de marcha del cono. Esto solo afecta el trazado de conos con vóxeles.}
\label{tab:performance_step}
\end{table}

Considerando como tiempos interactivos todo resultado por debajo de $33.3 ms$ o aproximadamente 30 cuadros por segundo. Nuestra implementación logra colocarse por debajo de este tiempo en todos los casos excepto los casos que comprenden resoluciones de volúmenes mayores a $256^3$.

\subsubsection{Densidad Geométrica y Velocidad de Voxelización.}
La cantidad de triángulos dentro del espacio que representa un vóxel afecta la velocidad de voxelización. Esto se debe a que este proceso requiere sincronización entre distintos hilos por fragmento. A mayor cantidad de triángulos en este espacio mayor es la cantidad de fragmentos generados por el proceso de rasterización. Cada uno de estos hilos debe esperar a escribir en la misma posición del volumen para garantizar atomicidad.

En la imagen \ref{fig:voxelization_times} se puede observar esta condición, especialmente en la escena Conference. Esta es una escena pequeña en escala, sin embargo de todas las escenas completas es la que posee mayor cantidad de triángulos como se puede observar en la tabla \ref{tab:scenes_attributes}. También es notable como disminuye el tiempo de voxelización para algunas escenas al aumentar la resolución del volumen y como aumenta al utilizar menores resoluciones, especialmente el salto entre $128^3$ a $64^3$.

\begin{figure}[H]
	\centering
	\begin{subfigure}{.49\textwidth}
		\centering
		\includegraphics[width=\linewidth]{media/voxelzation_64_cropped.pdf}
		\caption*{Resolución de volúmenes: $64^3$}
	\end{subfigure}
	\begin{subfigure}{.49\textwidth}
		\centering
		\includegraphics[width=\linewidth]{media/voxelzation_128_cropped.pdf}	
		\caption*{Resolución de volúmenes: $128^3$}
	\end{subfigure}
	\par\bigskip
	\begin{subfigure}{.49\textwidth}
		\centering
		\includegraphics[width=\linewidth]{media/voxelzation_256_cropped.pdf}
		\caption*{Resolución de volúmenes: $256^3$}
	\end{subfigure}
	\begin{subfigure}{.49\textwidth}
		\centering
		\includegraphics[width=\linewidth]{media/voxelzation_512_cropped.pdf}	
		\caption*{Resolución de volúmenes: $512^3$}
	\end{subfigure}
	\caption{Tiempos de voxelización dinámica y estática de escenas completas con distintas resoluciones para la representación en vóxeles.}
	\label{fig:voxelization_times}
\end{figure}

\subsubsection{Vacuidad y Velocidad de Trazado para la Iluminación Global de Vóxeles.}

En escenas donde existen muchos espacios vacíos el trazado de rayos suele tardar un poco más que en escenas densas de objetos. Esto se debe a que mientras la representación en vóxeles sea vacía en la posición que describe la apertura, dirección y punto de origen del cono, este cono debe seguir expandiéndose. A consecuencia mientras más espacios vacíos la representación en vóxeles tenga, más distancia tiene que recorrer cada cono aumentando la cantidad de operaciones de lectura sobre la representación en vóxeles.

Este problema se puede observar en la figura \ref{fig:gi_voxel_time} para las escenas Sponza y Cornell Box. La escena Cornell Box es en gran parte vacua con solo dos cuboides internos. La escena Sponza es densa en su interior por tanto la falta de objetos no es el principal problema. El problema de la Sponza reside en que la escena es larga y alta pero no ancha. Durante la voxelización cada triángulo se proyecta de forma ortogonal sobre cada eje para maximizar la voxelización, este frustum de proyección es cuadrado. Para las dimensiones de esta escena siempre quedara una cantidad considerable de vóxeles vacíos fuera de la geometría principal de la escena.

\begin{figure}[h]
	\centering
	\includegraphics[width=0.85\linewidth]{media/voxel_gi_time_cropped.pdf}
	\caption{Tiempos de cálculo de iluminación global sobre la representación en vóxeles para todas las escenas completas.}
	\label{fig:gi_voxel_time}
\end{figure}

\subsection{Trazado de Sombras y Volumen de Visibilidad}

\subsection{Apertura del Cono Especular}

\subsection{Apertura del Cono de Sombreado}
