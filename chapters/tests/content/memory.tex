\section{Estudio de Memoria}
Para examinar el consumo de memoria en la \ac{GPU} se utilizó la herramienta ProcExplorer. Nuestra implementación no utiliza una estructura de octree disperso por tanto se almacenan vóxeles sean estos vacíos o no. Esto implica un mayor uso de memoria. En la Tabla \ref{tab:memory} se puede observar la memoria reservada por resolución de la representación en vóxeles. Estos valores son constantes sobre cualquier escenario en la aplicación. En la Tabla \ref{tab:memory} se pueden observar los resultados.
\begin{table}[H]
	\centering
	\begin{tabular}{ll}
	Resolución de la Representación en Vóxeles & Memoria Reservada             \\ \hline
	\multicolumn{1}{|l|}{$512^3$ vóxeles}                  & \multicolumn{1}{l|}{2800 MB}  \\
	\multicolumn{1}{|l|}{$256^3$ vóxeles}                  & \multicolumn{1}{l|}{540.3 MB} \\
	\multicolumn{1}{|l|}{$128^3$ vóxeles}                  & \multicolumn{1}{l|}{253 MB}   \\
	\multicolumn{1}{|l|}{$64^3$ vóxeles}                   & \multicolumn{1}{l|}{216.7 MB} \\ \hline
	\end{tabular}
	\caption{Consumo de memoria en la \ac{GPU} para distintas resoluciones de la representación en vóxeles.}
	\label{tab:memory}
\end{table}

