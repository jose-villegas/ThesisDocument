\section{Estructura Jerárquica} % (fold)
\label{sec:estructura_jerarquica}
Durante el trazado de conos se utilizan distintos niveles de detalle de la escena voxelizada a medida que el diámetro del cono se expande por su recorrido en escena. En el trabajo de Crassin estos niveles de detalle se construyen utilizando la profundidad del \emph{octree} disperso, donde el nodo raíz es el nivel de detalle más bajo y las hojas del árbol contienen el máximo nivel detalle. El proceso de filtrado desde las hojas al nodo raíz fue explicado en la sección \ref{subsub:mipmaping_orig}. En nuestra implementación con texturas 3D esto representa simplemente los distintos niveles de \emph{mipmapping} en una textura, estos pueden ser generados con una invocación al método \emph{glGenerateMipmap} en OpenGL. Una ventaja de utilizar texturas 3D es que el filtrado cuadrilineal es soportado de forma nativa por hardware, sin necesidad de construir bloques por cada vóxel como se explica en la sección \ref{subsub:voxelcontent_orig}. Esto simplifica de gran manera la construcción de la estructura jerárquica.
% section estructura_jerarquica (end)

\subsection{Filtrado Anisotrópico de Vóxeles} % (fold)
\label{sub:mipmapping_direccioanl}
Es posible obtener resultados más precisos durante el trazado de conos utilizando vóxeles direccionales o anisótropicos. Como fue explicada la generación de los niveles de detalle en la sección anterior solo se obtienen vóxeles isotrópicos, esto quiere decir que poseen el mismo valor sin importar la dirección como son observados. Los problemas que puede ocasionar esta forma de representar los niveles de detalles, fueron explicados en la sección \ref{subsub:aniso_voxels_orig}. Implementar filtrado direccional consiste en que cada vóxel debe almacenar seis canales, uno por cada eje direccional positivo y negativo. En nuestra implementación con texturas 3D esto se traduce en seis texturas 3D (una por cada dirección) a la mitad de la resolución del volumen original. Para realizar el filtrado anisotrópico de alto rendimiento se utiliza \emph{compute shaders}.
% section mipmapping_direccioanl (end)
